\documentclass[final,hyperref={pdfpagelabels=false}]{beamer}
\mode<presentation>
  {
  %  \usetheme{Berlin}
  \usetheme{USI}
  }
  %\usepackage{times}
  \usepackage{amsmath,amsthm, amssymb, amsfonts, latexsym}
  \boldmath
  \usepackage[english]{babel}
  \usepackage[latin1]{inputenc}
  \usepackage[orientation=portrait,size=a0,scale=1.0,debug]{beamerposter}
  \usepackage{verbatim}
  \usepackage{IEEEtrantools}
  \usepackage{tikz}
  \usetikzlibrary{calc}

  \newcommand{\gearmacro}[5]{%  
    \foreach \i in {1,...,#1} {%
      [rotate=(\i-1)*360/#1]  (0:#2)  arc (0:#4:#2) {[rounded corners=1.5pt]
        -- (#4+#5:#3)  arc (#4+#5:360/#1-#5:#3)} --  (360/#1:#2)
    }}  

  %%%%%%%%%%%%%%%%%%%%%%%%%%%%%%%%%%%%%%%%%%%%%%%%%%%%%%%%%%%%%%%%%%%%%%%%%%%%%%%%%5
  \graphicspath{{figures/}}
  \title[Crypto History]{The History of Cryptography}
  \author[Hansen and Wolf]{Arne Hansen and Prof Stefan Wolf}
  \institute[USI]{Cryptography and Quantum Information, USI Lugano}
  \date{Jul. 31th, 2007}


  %%%%%%%%%%%%%%%%%%%%%%%%%%%%%%%%%%%%%%%%%%%%%%%%%%%%%%%%%%%%%%%%%%%%%%%%%%%%%%%%%5
  \begin{document}
  \tikzset{
    TLstegan/.style = {fill = gray!20!white,
                      draw=AHdarkblue,align= left,text width=5cm,font=\tiny,line width=2pt},
    TLcrypto/.style = {fill = gray!20!white,
                      draw=AHdarkorange,align= left,text width=5cm,font=\tiny,line width=2pt},
    TLphases/.style = {color=black,left,text width=10cm,font=\tiny,line width=0pt}
  }
  \begin{frame}{} 
    \vfill
    \vfill
    \begin{block}{\large Timeline of Cryptography}
      \begin{figure}
      \centering
      \begin{tikzpicture}[scale=1.9]
          % define coordinates
          \coordinate (Herodotus) at (-5,0);
          \coordinate (Caesar) at (-0.6,0);
          \coordinate (scytale) at (-4.5,0);
          \coordinate (freq-analysis) at (9.5,0);
          \coordinate (Vigenere) at (15.53,0);
          \coordinate (Babbage) at (18.50,0); 
          % forall coordinates in 20th century: magnify by factor 10
          \coordinate (one-time-pad) at (1.7,0);
          \coordinate (enigma) at (1.8,0);
          \coordinate (rejewski) at (3.2,0);
          \coordinate (navajo) at (4.2,0);
          \coordinate (shannon) at (4.8,0);
          \coordinate (colossus) at (4.3,0);
          \coordinate (nsa) at (5.2,0);
          \coordinate (feistel) at (7.1,0);
          \coordinate (des) at (7.6,0);
          \coordinate (diffie-hellman) at (7.6,0);
          \coordinate (rsa) at (7.7,0);
          \coordinate (BB84) at (8.4,0);
          \coordinate (pgp) at (9.1,0); 
          % phases of cryptography
          \fill [AHdarkorange!30] ($(Caesar) + (0.0,-8)$) rectangle (30,-7);
          \draw[line width=3pt, color=AHdarkorange] (Caesar) -- ($(Caesar) + (0.0,-8)$) node[TLphases, above right] {monoalphabetic substitution cipher};
          \draw[line width=3pt, color=AHdarkorange] (Vigenere) -- ($(Vigenere) + (0.0,-8)$) node[TLphases, above right] {{\em poly}alphabetic subst. cipher};
          \draw[line width=3pt, color=AHdarkorange] ($(enigma) + (19,0)$) -- ($(enigma) + (19.0,-8)$) node[TLphases, above right] {polyalphabetic {\em machine} ciphers};
          % create baseline
          \draw[line width=2pt,dashed] (-10,0) -- (-5.20,0) ;
          \draw[line width=2pt] (-5.20,0) -- (32,0);
          \draw[line width=5pt] (18.8,0) -- (31.2,0);
          \coordinate (cryptography) at (-9.8,3);
          \coordinate (cryptanalysis) at (-9.8,-3);
          \draw[line width=6pt,color=black!80] (25,0) circle (6.2); 
          \node[rotate=90] (label) at (cryptography) {\small \textsc{Cryptography}};
          \node[rotate=90] (label) at (cryptanalysis) {\small \textsc{Cryptanalysis}};
          \foreach \t in {-5,...,18}{
            \draw[line width=2pt] (\t,-.2) -- (\t,.2) node [below=.5cm] {\tiny \t 00};
            \draw[line width=1pt] ($(\t,-.1) + (.5,0)$) -- ($(\t,.1) + (0.5,0)$);}
          \foreach \t in {0,...,12}{
            \draw[line width=3pt] let \n1 = {int(1900+\t*10)} in
              ($(19,0)+(\t,-.2)$) -- ($(19,0)+(\t,.2)$) node [below=.6cm] {\tiny \n1};
            \draw[line width=2pt] ($(19.5,0)+(\t,-.1)$) -- ($(19.5,0)+(\t,.1)$);}
            
          % declare nodes
          % cryptographic achievement
          \draw[line width=1pt] ($(Herodotus) + (0,0.0)$) -- ($(Herodotus) + (0,1.0)$) node [TLstegan,above left] {Greek historian {\em Herodotus} tells about steganography};
          \draw[line width=1pt] ($(scytale) + (0,0.0)$) -- ($(scytale) + (0,4.0)$) node [TLcrypto,above] {\textbf{ 5th century BC} {\em Scytale}: cryptographic device used by the Spartans};
          \draw[line width=1pt] ($(Caesar) + (0,0.0)$) -- ($(Caesar) + (0,1.0)$) node [TLcrypto,above right] {{\em Caesar shift cipher}: encypher by shifting letters of the alphabet};
          \draw[line width=1pt] ($(Vigenere) + (0,0.0)$) -- ($(Vigenere) + (0,1.0)$) node [TLcrypto,above left] {\textbf{1553} Giovan Battista Bellaso develops the poly-alphabetic so-called {\em Vigenere cipher}};
          \draw[line width=1pt] ($(one-time-pad) + (19,0.0)$) -- ($(one-time-pad) + (19,1.0)$) node [TLcrypto,above left] {\textbf{1917} the {\em one-time-pad} --- a cipher with absolute security --- is invented};
          \draw[line width=1pt] ($(enigma) + (19,0.0)$) -- ($(enigma) + (19,3.0)$) node [TLcrypto,above] {\textbf{1918}{\em ENIGMA} invented by Arthur Scherbius};
          \draw[line width=1pt] ($(shannon) + (19,0.0)$) -- ($(shannon) + (19,2.3)$) node [TLcrypto,above] {\textbf{1948/49} {\em Claude Shannon} creates the basis for information theory and formal cryptography};
          \draw[line width=1pt] ($(navajo) + (19,0.0)$) -- ($(navajo) + (19,1.0)$) node [TLcrypto,above] {\textbf{1942} {\em Navajos} join the US army to translate, i.e. encipher messages};
          \draw[line width=1pt] ($(feistel) + (19,0.0)$) -- ($(feistel) + (19,7.0)$) node [TLcrypto,above left] {\textbf{1971} Horst Feistel develops the block cipher {\em Lucifer} for IBM};
          \draw[line width=1pt] ($(BB84) + (19,0.0)$) -- ($(BB84) + (19,7.0)$) node [TLcrypto,above right] {\textbf{1984} Charles Bennet and Gilles Brassard invent a protocol for quantum key distribution, referred to as {\em BB84}};
          \draw[line width=1pt] ($(des) + (19,0.0)$) -- ($(des) + (19,4.0)$) node [TLcrypto,above left] {\textbf{1976} A version of Lucifer is made the {\em Data Encryption Standard (DES)}};
          \draw[line width=1pt] ($(rsa) + (19,0.0)$) -- ($(rsa) + (19,4.0)$) node [TLcrypto,above right] {\textbf{1977} Ron Rivest, Adi Shamir and Leonard Aldeman develop an asymmetric {\em public key crypto} algorithm {\em RSA} (as proposed by Diffie)};
          \draw[line width=1pt] ($(diffie-hellman) + (19,0.0)$) -- ($(diffie-hellman) + (19,0.5)$) node [TLcrypto,above] {\textbf{1976} Martin Hellman, Whitfield Diffie and Ralph Merkle publish a protocol for  {\em cryptographic key exchange}};
          \draw[line width=1pt] ($(pgp) + (19,0.0)$) -- ($(pgp) + (19,0.5)$) node [TLcrypto,above right] {\textbf{1991} Phil Zimmermann compiles symmetric, antisymmetric and signing algorithms to a bundle {\em Pretty Good Privacy} intended for broader public use};
          % cryptoanalytic achievements
          \draw[line width=1pt] ($(freq-analysis) + (0,0.)$) -- ($(freq-analysis) + (0,-1.0)$) node [TLcrypto,below left] {{\em frequency-analysis}: of monoalphabetic ciphers};
          \draw[line width=1pt] ($(Babbage) + (0,0.)$) -- ($(Babbage) + (0,-1.0)$) node [TLcrypto,below left] {\textbf{1850s} {\em Charles Babbage} breaks polyalphabetic ciphers};
          \draw[line width=1pt] ($(rejewski) + (19,0)$) -- ($(rejewski) + (19,-3.0)$) node [TLcrypto,below left] {\textbf{1932} {\em Marian Rejewski} breaks the Enigma};
          \draw[line width=1pt] ($(colossus) + (19,0)$) -- ($(colossus) + (19,-1.0)$) node [TLcrypto,below left] {\textbf{1943} Tommy Flower implements Max Newman's {\em Colossus}, the first modern computer};
          \draw[line width=1pt] ($(nsa) + (19,0)$) -- ($(nsa) + (19,-1.0)$) node [TLcrypto,below right] {\textbf{1952} the {\em National Security Agency (NSA)} is found};
      \end{tikzpicture}
      \end{figure}
      The timeline shows the greatest achievements in the history of cryptography and cryptanalysis. On the bottom the evolution from monoalphabetic substitution ciphers to current computer based polyalphabetic substition ciphers is drawn.
    \end{block}
  \end{document}
