\documentclass[final,hyperref={pdfpagelabels=false}]{beamer}
\mode<presentation>
  {
  %  \usetheme{Berlin}
  \usetheme{USI}
  }
  %\usepackage{times}
  \usepackage{amsmath,amsthm, amssymb, amsfonts, latexsym}
  \boldmath
  \usepackage[english]{babel}
  \usepackage[latin1]{inputenc}
  \usepackage[orientation=portrait,size=a0,scale=1.4,debug]{beamerposter}
  \usepackage{verbatim}
  \usepackage{IEEEtrantools}
  \usepackage{tikz}
  \usetikzlibrary{calc}

  \newcommand{\gearmacro}[5]{%  
    \foreach \i in {1,...,#1} {%
      [rotate=(\i-1)*360/#1]  (0:#2)  arc (0:#4:#2) {[rounded corners=1.5pt]
        -- (#4+#5:#3)  arc (#4+#5:360/#1-#5:#3)} --  (360/#1:#2)
    }}  

  %%%%%%%%%%%%%%%%%%%%%%%%%%%%%%%%%%%%%%%%%%%%%%%%%%%%%%%%%%%%%%%%%%%%%%%%%%%%%%%%%5
  \graphicspath{{figures/}}
  \title[Crypto History]{The History of Cryptography}
  \author[Hansen and Wolf]{Arne Hansen and Prof Stefan Wolf}
  \institute[USI]{Cryptography and Quantum Information, USI Lugano}
  \date{Jul. 31th, 2007}


  %%%%%%%%%%%%%%%%%%%%%%%%%%%%%%%%%%%%%%%%%%%%%%%%%%%%%%%%%%%%%%%%%%%%%%%%%%%%%%%%%5
  \begin{document}
  \tikzset{
    TLstegan/.style = {fill = gray!20!white,
                      draw=AHdarkblue,align= left,text width=6cm,font=\tiny,line width=2pt},
    TLcrypto/.style = {fill = gray!20!white,
                      draw=AHdarkorange,align= left,text width=6cm,font=\tiny,line width=2pt}
  }
  \begin{frame}{} 
    \vfill
    \begin{columns}[t]
    \begin{column}{.48\linewidth}
    \begin{block}{\large Hiding Information: A Definition of Cryptography}
      \begin{figure}
      \centering
      \begin{tikzpicture}[font=\footnotesize,
          grow=right, level 1/.style={sibling distance=6em},
          level 2/.style={sibling distance=6em}, level distance=10cm]
          \node {Hidding Information} % root
            child { node {Steganography: writing physically hidden}
            }
            child { node {\textbf{Cryptography} information hidden}
              child { node {Substitution} }
              child { node {Permutation} }
            };
      \end{tikzpicture}
      \caption{Categories of Secrecy}
      \end{figure}
      \alert{Cryptography} hiding content of a message without hiding the writing itself. \\
      \alert{Steganography} physically hiding the message (invisible ink, \ldots)
    \end{block}
    \begin{block}{\large The Scheme of Cryptography}
    \begin{figure}
      \begin{tikzpicture}[scale=0.8]
      \coordinate (plaintext1) at (-20.,0);
      \coordinate (plaintext2) at (20.,0);
      \coordinate (key1) at (-11.,8);
      \coordinate (key2) at (9,8);
      \coordinate (cipher_algorithm) at (-10,0);
      \coordinate (decipher_algorithm) at (10,0);
      \coordinate (ciphertext) at (0,0);
      % drawing the text sheets
      \filldraw[very thick,color=blue!70!black!90, fill=blue!50!black!50!] plot[smooth cycle,tension=0.05] coordinates{($(ciphertext)+ (-1.4,-2.2)$) ($(ciphertext)+ (-1.4,2.2)$) ($(ciphertext)+ (1.4,2.2)$) ($(ciphertext)+ (1.4,-2.2)$)};
       \foreach \y in {-4,...,4}
                 \draw[dashed] ($(ciphertext)+(-1.2,0.5*\y)$)--($(ciphertext)+(1.2,0.5*\y)$);
      \filldraw[very thick,color=red!70!black!90, fill=red!50!black!50!] plot[smooth cycle,tension=0.05] coordinates{($(plaintext1)+ (-1.4,-2.2)$) ($(plaintext1)+ (-1.4,2.2)$) ($(plaintext1)+ (1.4,2.2)$) ($(plaintext1)+ (1.4,-2.2)$)};
       \foreach \y in {-4,...,4}
                 \draw ($(plaintext1)+(-1.2,0.5*\y)$)--($(plaintext1)+(1.2,0.5*\y)$);
      \filldraw[very thick,color=red!70!black!90, fill=red!50!black!50!] plot[smooth cycle,tension=0.05] coordinates{($(plaintext2)+ (-1.4,-2.2)$) ($(plaintext2)+ (-1.4,2.2)$) ($(plaintext2)+ (1.4,2.2)$) ($(plaintext2)+ (1.4,-2.2)$)};
       \foreach \y in {-4,...,4}
                 \draw ($(plaintext2)+(-1.2,0.5*\y)$)--($(plaintext2)+(1.2,0.5*\y)$);
      % drawing the keys
      \draw[line width=2pt] (key1) circle (0.5);
      \draw[line width=2pt] ($(key1) + (0.5,0)$) -- ($(key1) + (2,0)$);
      \filldraw plot coordinates{($(key1) + (1.6,0)$) ($(key1) + (2,0)$) ($(key1) + (2,-0.5)$) ($(key1) + (1.6,-0.5)$) };
      \draw[line width=2pt] (key2) circle (0.5);
      \draw[line width=2pt] ($(key2) + (0.5,0)$) -- ($(key2) + (2,0)$);
      \filldraw plot coordinates{($(key2) + (1.6,0)$) ($(key2) + (2,0)$) ($(key2) + (2,-0.5)$) ($(key2) + (1.6,-0.5)$) };
      % drawing the cipher algorithms
      \filldraw[very thick,color=black!70!black!90, fill=black!50!black!50!] plot[smooth cycle,tension=0.05] coordinates{($(cipher_algorithm)+ (-1.4,-1.4)$) ($(cipher_algorithm)+ (-1.4,1.4)$) ($(cipher_algorithm)+ (1.4,1.4)$) ($(cipher_algorithm)+ (1.4,-1.4)$)};
      \node [label={[white] above:\tiny $>>>>$}] (label) at (cipher_algorithm) {};
      \node [label={[white] below:\tiny $<<<<$}] (label) at (cipher_algorithm) {};
      \filldraw[very thick,color=black!70!black!90, fill=black!50!black!50!] plot[smooth cycle,tension=0.05] coordinates{($(decipher_algorithm)+ (-1.4,-1.4)$) ($(decipher_algorithm)+ (-1.4,1.4)$) ($(decipher_algorithm)+ (1.4,1.4)$) ($(decipher_algorithm)+ (1.4,-1.4)$)};
      \node [label={[white] above:\tiny $>>>>$}] (label) at (decipher_algorithm) {};
      \node [label={[white] below:\tiny $<<<<$}] (label) at (decipher_algorithm) {};
      %\draw[fill=white] \gearmacro{8}{2}{2.4}{20}{2};
      % drawing the arrows
      \draw[line width=2pt, ->] ($(key1) + (+1,-1.5)$) -- ($(cipher_algorithm) + (0,3)$);
      \draw[line width=2pt, ->] ($(plaintext1) + (3.,0)$) -- ($(cipher_algorithm) + (-3,0)$);
      \draw[line width=2pt, ->] ($(cipher_algorithm) + (3,0.)$) -- ($(ciphertext) + (-3.,0.)$);
      \draw[line width=2pt, ->] ($(ciphertext) + (3.,0.)$) -- ($(decipher_algorithm) + (-3,0.)$);
      \draw[line width=2pt, <-] ($(decipher_algorithm) + (0,3)$) -- ($(key2) + (+1,-1.5)$);
      \draw[line width=2pt, ->] ($(decipher_algorithm) + (3,0)$) -- ($(plaintext2) + (-3.,0)$);
      %\node [label=right:$\S$] (label) at (sep) {};
      % \node [label=right:$\PPT$] (label) at (ppt) {};
      %\node [label=right:$\rho_{AB}$] (label) at (rho) {};
      %\node [label=right:$\mathcal{D}$] (label) at (dens) {};
      \end{tikzpicture}
    \end{figure}
    \end{block}
    \end{column}
    \begin{column}{.48\linewidth}
      \begin{block}{\large The neverending competition between Cryptographers and Cryptanalysts}
        Cryptanalysts study cryptographic systems and try to break them. They search for ways to access the hidden information. In history the advantage alternated between cryptographers and cryptanalysists. 
        \par First cryptographic monoalphabetic substitution ciphers --- i.e. ciphers built on replacing letters according to a fixed scheme like the Caesar cipher --- were safe as long the substitution scheme was kept secret. The first cryptanalytic breakthrough stemmed from linguistic studies of the Koran in the $9^{th}$ century in the Arabic world. Theologians analyzed the structure of text in order to determine their origin, thereby counting letters and studying the frequencies with which they appeared. It turned out: some letters are used more often than others. In English for example the most frequent letter is ``e''. Counting the frequencies of letters in a ciphertext makes it pretty easy to guess the replacement scheme used to encrypt a message. Once {\em frequency analysis} was developed cryptanalysists could basically break any message until cryptographic methods were developed further.
        \par One might think, that after the one-time pad was shown to be absolutely secure, the competition might have been settled in favor of cryptographers. Unfortunately the one-time pad is not efficient as the key (that has to be destributed secretely) has to be as long as the message itself.
      \end{block}
    \end{column}
    \end{columns}
    \vfill
    \vfill
    \begin{block}{\large Timeline of Cryptography}
      \begin{figure}
      \centering
      \begin{tikzpicture}[scale=1.9]
          % create baseline
          \draw[line width=2pt,dashed] (-10,0) -- (-5.20,0) ;
          \draw[line width=2pt] (-5.20,0) -- (20.55,0);
          \coordinate (cryptography) at (-9.8,3);
          \coordinate (cryptanalysis) at (-9.8,-3);
          \node[rotate=90] (label) at (cryptography) {\small \textsc{Cryptography}};
          \node[rotate=90] (label) at (cryptanalysis) {\small \textsc{Cryptanalysis}};
          \foreach \t in {-5,...,20}{
            \draw[line width=2pt] (\t,-.2) -- (\t,.2) node [below=.4cm] {\tiny \t 00};
            \draw[line width=1pt] ($(\t,-.1) + (.5,0)$) -- ($(\t,.1) + (0.5,0)$);}
          % define coordinates
          \coordinate (Herodotus) at (-5,0);
          \coordinate (Caesar) at (-0.6,0);
          \coordinate (scytale) at (-4.5,0);
          \coordinate (freq-analysis) at (9.5,0);
          \coordinate (Vigenere) at (15.53,0);
          \coordinate (Babbage) at (18.50,0); 
          \coordinate (one-time-pad) at (19.17,0);
          \coordinate (enigma) at (19.18,0);
          % declare nodes
          % cryptographic achievement
          \draw[line width=1pt] ($(Herodotus) + (0,0.2)$) -- ($(Herodotus) + (0,1.0)$) node [TLstegan,above left] {Greek historian {\em Herodotus} tells about steganography};
          \draw[line width=1pt] ($(scytale) + (0,0.2)$) -- ($(scytale) + (0,4.0)$) node [TLcrypto,above] {\textbf{ 5th century BC} {\em Scytale}: cryptographic device used by the Spartans};
          \draw[line width=1pt] ($(Caesar) + (0,0.2)$) -- ($(Caesar) + (0,1.0)$) node [TLcrypto,above right] {{\em Caesar shift cipher}: encypher by shifting letters of the alphabet};
          \draw[line width=1pt] ($(Vigenere) + (0,0.2)$) -- ($(Vigenere) + (0,1.0)$) node [TLcrypto,above left] {\textbf{1553} Giovan Battista Bellaso develops the poly-alphabetic so-called {\em Vigenere cipher}};
          \draw[line width=1pt] ($(one-time-pad) + (0,0.2)$) -- ($(one-time-pad) + (0,3.0)$) node [TLcrypto,above left] {\textbf{1917} the {\em one-time-pad} --- a cipher with absolute security --- is invented};
          \draw[line width=1pt] ($(enigma) + (0,0.2)$) -- ($(enigma) + (0,1.0)$) node [TLcrypto,above right] {\textbf{1918}{\em ENIGMA} invented by Arthur Scherbius};
          % cryptoanalytic achievements
          \draw[line width=1pt] ($(freq-analysis) + (0,-0.5)$) -- ($(freq-analysis) + (0,-1.0)$) node [TLcrypto,below left] {{\em frequency-analysis}: of monoalphabetic ciphers};
          \draw[line width=1pt] ($(Babbage) + (0,-0.5)$) -- ($(Babbage) + (0,-1.0)$) node [TLcrypto,below left] {\textbf{1850s} {\em Charles Babbage} breaks polyalphabetic ciphers};
      \end{tikzpicture}
      \end{figure}
    \end{block}
    \vfill
    \begin{columns}[t]
      \begin{column}{.48\linewidth}
        \begin{block}{Scytale: a transposition cipher}
        The message ``Help me, I am under attack'' is written on the scytale in rows
        \begin{equation*}
        \begin{array}{|c|c|c|c|c|}
              &    &    &    &   \\
           H  & E  & L  & P  & M \\
           E  & I  & A  & M  & U \\
           N  & D  & E  & R  & A \\
           T  & T  & A  & C  & K \\   
              &    &    &    &   
        \end{array}
        \end{equation*}
        After unwinding the band it becomes scrambled to
        \begin{equation*}
          HENTEIDTLAEAPMRCMUAK
        \end{equation*}
        \end{block}
      \end{column}
      \begin{column}{.48\linewidth}
        \begin{block}{The Caesar Shift Cipher}
        \end{block}

        \begin{block}{Frequency Analysis}
        \end{block}

        \begin{block}{The One-Time Pad}
        \end{block}
        \begin{block}{Enigma}
        \end{block}
      \end{column}
    \end{columns}
  \end{frame}
\end{document}


%%%%%%%%%%%%%%%%%%%%%%%%%%%%%%%%%%%%%%%%%%%%%%%%%%%%%%%%%%%%%%%%%%%%%%%%%%%%%%%%%%%%%%%%%%%%%%%%%%%%
%%% Local Variables: 
%%% mode: latex
%%% TeX-PDF-mode: t
%%% End:
