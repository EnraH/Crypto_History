\documentclass[final,hyperref={pdfpagelabels=false}]{beamer}
\mode<presentation>
  {
  %  \usetheme{Berlin}
  \usetheme{USI}
  }
  %\usepackage{times}
  \usepackage{amsmath,amsthm, amssymb, amsfonts, latexsym}
  \boldmath
  \usepackage[english]{babel}
  \usepackage[latin1]{inputenc}
  \usepackage[orientation=portrait,size=a0,scale=1.4,debug]{beamerposter}
  \usepackage{verbatim}
  \usepackage{IEEEtrantools}
  \usepackage{tikz}
  \usetikzlibrary{calc}
  \usefonttheme{professionalfonts}

  \newcommand{\gearmacro}[5]{%  
    \foreach \i in {1,...,#1} {%
      [rotate=(\i-1)*360/#1]  (0:#2)  arc (0:#4:#2) {[rounded corners=1.5pt]
        -- (#4+#5:#3)  arc (#4+#5:360/#1-#5:#3)} --  (360/#1:#2)
    }}  

  %%%%%%%%%%%%%%%%%%%%%%%%%%%%%%%%%%%%%%%%%%%%%%%%%%%%%%%%%%%%%%%%%%%%%%%%%%%%%%%%%5
  \graphicspath{{figures/}}
  \title[Crypto History]{The History of Cryptography}
  \author[Hansen and Wolf]{Arne Hansen and Prof Stefan Wolf}
  \institute[USI]{Cryptography and Quantum Information, USI Lugano}
  \date{Jul. 31th, 2007}


  %%%%%%%%%%%%%%%%%%%%%%%%%%%%%%%%%%%%%%%%%%%%%%%%%%%%%%%%%%%%%%%%%%%%%%%%%%%%%%%%%5
  \begin{document}
  \tikzset{
    TLstegan/.style = {fill = gray!20!white,
                      draw=AHdarkblue,align= left,text width=6cm,font=\tiny,line width=2pt},
    TLcrypto/.style = {fill = gray!20!white,
                      draw=AHdarkorange,align= left,text width=6cm,font=\tiny,line width=2pt}
  }
  \begin{frame}{} 
    \vfill
    \begin{columns}[t]
    \begin{column}{.48\linewidth}
    \begin{block}{\large Hiding Information: A Definition of Cryptography}
      \begin{figure}
      \centering
      \begin{tikzpicture}[font=\footnotesize,
          grow=right, level 1/.style={sibling distance=6em},
          level 2/.style={sibling distance=6em}, level distance=10cm]
          \node {Hidding Information} % root
            child { node {Steganography: writing physically hidden}
            }
            child { node {\textbf{Cryptography} information hidden}
              child { node {Substitution} }
              child { node {Permutation} }
            };
      \end{tikzpicture}
      \caption{Categories of Secrecy}
      \end{figure}
      \alert{Cryptography} hidding content of a message without hiding the writing itself.
    \end{block}
    \begin{block}{\large The Scheme of Cryptography}
    \begin{figure}
      \begin{tikzpicture}[scale=0.8]
      \coordinate (plaintext1) at (-20.,0);
      \coordinate (plaintext2) at (20.,0);
      \coordinate (key1) at (-11.,8);
      \coordinate (key2) at (9,8);
      \coordinate (cipher_algorithm) at (-10,0);
      \coordinate (decipher_algorithm) at (10,0);
      \coordinate (ciphertext) at (0,0);
      % drawing the text sheets
      \filldraw[very thick,color=blue!70!black!90, fill=blue!50!black!50!] plot[smooth cycle,tension=0.05] coordinates{($(ciphertext)+ (-1.4,-2.2)$) ($(ciphertext)+ (-1.4,2.2)$) ($(ciphertext)+ (1.4,2.2)$) ($(ciphertext)+ (1.4,-2.2)$)};
       \foreach \y in {-4,...,4}
                 \draw[dashed] ($(ciphertext)+(-1.2,0.5*\y)$)--($(ciphertext)+(1.2,0.5*\y)$);
      \filldraw[very thick,color=red!70!black!90, fill=red!50!black!50!] plot[smooth cycle,tension=0.05] coordinates{($(plaintext1)+ (-1.4,-2.2)$) ($(plaintext1)+ (-1.4,2.2)$) ($(plaintext1)+ (1.4,2.2)$) ($(plaintext1)+ (1.4,-2.2)$)};
       \foreach \y in {-4,...,4}
                 \draw ($(plaintext1)+(-1.2,0.5*\y)$)--($(plaintext1)+(1.2,0.5*\y)$);
      \filldraw[very thick,color=red!70!black!90, fill=red!50!black!50!] plot[smooth cycle,tension=0.05] coordinates{($(plaintext2)+ (-1.4,-2.2)$) ($(plaintext2)+ (-1.4,2.2)$) ($(plaintext2)+ (1.4,2.2)$) ($(plaintext2)+ (1.4,-2.2)$)};
       \foreach \y in {-4,...,4}
                 \draw ($(plaintext2)+(-1.2,0.5*\y)$)--($(plaintext2)+(1.2,0.5*\y)$);
      % drawing the keys
      \draw[line width=2pt] (key1) circle (0.5);
      \draw[line width=2pt] ($(key1) + (0.5,0)$) -- ($(key1) + (2,0)$);
      \filldraw plot coordinates{($(key1) + (1.6,0)$) ($(key1) + (2,0)$) ($(key1) + (2,-0.5)$) ($(key1) + (1.6,-0.5)$) };
      \draw[line width=2pt] (key2) circle (0.5);
      \draw[line width=2pt] ($(key2) + (0.5,0)$) -- ($(key2) + (2,0)$);
      \filldraw plot coordinates{($(key2) + (1.6,0)$) ($(key2) + (2,0)$) ($(key2) + (2,-0.5)$) ($(key2) + (1.6,-0.5)$) };
      % drawing the cipher algorithms
      \filldraw[very thick,color=black!70!black!90, fill=black!50!black!50!] plot[smooth cycle,tension=0.05] coordinates{($(cipher_algorithm)+ (-1.4,-1.4)$) ($(cipher_algorithm)+ (-1.4,1.4)$) ($(cipher_algorithm)+ (1.4,1.4)$) ($(cipher_algorithm)+ (1.4,-1.4)$)};
      \node [label={[white] above:\tiny $>>>>$}] (label) at (cipher_algorithm) {};
      \node [label={[white] below:\tiny $<<<<$}] (label) at (cipher_algorithm) {};
      \filldraw[very thick,color=black!70!black!90, fill=black!50!black!50!] plot[smooth cycle,tension=0.05] coordinates{($(decipher_algorithm)+ (-1.4,-1.4)$) ($(decipher_algorithm)+ (-1.4,1.4)$) ($(decipher_algorithm)+ (1.4,1.4)$) ($(decipher_algorithm)+ (1.4,-1.4)$)};
      \node [label={[white] above:\tiny $>>>>$}] (label) at (decipher_algorithm) {};
      \node [label={[white] below:\tiny $<<<<$}] (label) at (decipher_algorithm) {};
      %\draw[fill=white] \gearmacro{8}{2}{2.4}{20}{2};
      % drawing the arrows
      \draw[line width=2pt, ->] ($(key1) + (+1,-1.5)$) -- ($(cipher_algorithm) + (0,3)$);
      \draw[line width=2pt, ->] ($(plaintext1) + (3.,0)$) -- ($(cipher_algorithm) + (-3,0)$);
      \draw[line width=2pt, ->] ($(cipher_algorithm) + (3,0.)$) -- ($(ciphertext) + (-3.,0.)$);
      \draw[line width=2pt, ->] ($(ciphertext) + (3.,0.)$) -- ($(decipher_algorithm) + (-3,0.)$);
      \draw[line width=2pt, <-] ($(decipher_algorithm) + (0,3)$) -- ($(key2) + (+1,-1.5)$);
      \draw[line width=2pt, ->] ($(decipher_algorithm) + (3,0)$) -- ($(plaintext2) + (-3.,0)$);
      %\node [label=right:$\S$] (label) at (sep) {};
      % \node [label=right:$\PPT$] (label) at (ppt) {};
      %\node [label=right:$\rho_{AB}$] (label) at (rho) {};
      %\node [label=right:$\mathcal{D}$] (label) at (dens) {};
      \end{tikzpicture}
    \end{figure}
    \end{block}
    \end{column}
    \begin{column}{.48\linewidth}
      \begin{block}{\large A battle between Cryptographers and Cryptoanalysts}
        An encrypted message remains only secret if the there is no way to break the restore the the 
      \end{block}
    \end{column}
    \end{columns}
    \vfill
    \vfill
    \begin{block}{\large Timeline of Cryptography}
      \begin{figure}
      \centering
      \begin{tikzpicture}[scale=1.8]
          % create baseline
          \draw[line width=2pt] (-5.20,0) -- (20.55,0);
          \foreach \t in {-5,...,20}{
            \draw[line width=2pt] (\t,-.2) -- (\t,.2) node [below=.4cm] {\tiny \t 00};
            \draw[line width=1pt] ($(\t,-.1) + (.5,0)$) -- ($(\t,.1) + (0.5,0)$);}
          % define coordinates
          \coordinate (Herodotus) at (-5,0);
          \coordinate (Caesar) at (-0.6,0);
          \coordinate (scytale) at (-4.5,0);
          \coordinate (freq-analysis) at (9.5,0);
          % declare nodes
          \draw[line width=1pt] ($(Herodotus) + (0,0.3)$) -- ($(Herodotus) + (0,3.0)$) node [TLstegan,below left] {Greek historian {\em Herodotus} tells about steganography};
          \draw[line width=1pt] ($(scytale) + (0,0.3)$) -- ($(scytale) + (0,4.0)$) node [TLcrypto,above] {\textbf{ 5th century BC} {\em Scytale}: cryptographic device used by the Spartans};
          \draw[line width=1pt] ($(Caesar) + (0,0.3)$) -- ($(Caesar) + (0,3.0)$) node [TLcrypto,below right] {{\em Caesar shift cipher}: encypher by shifting letters of the alphabet};
          \draw[line width=1pt] ($(freq-analysis) + (0,0.3)$) -- ($(freq-analysis) + (0,-3.0)$) node [TLcrypto,above left] {{\em frequency-analysis}: of monoalphabetic ciphers};
      \end{tikzpicture}
      \end{figure}
    \end{block}
    \vfill
    \begin{columns}[t]
      \begin{column}{.48\linewidth}
        \begin{block}{Scytale: a transposition cipher}
        The message ``Help me, I am under attack'' is written on the scytale in rows
        \begin{equation*}
        \begin{array}{|c|c|c|c|c|}
              &    &    &    &   \\
           H  & E  & L  & P  & M \\
           E  & I  & A  & M  & U \\
           N  & D  & E  & R  & A \\
           T  & T  & A  & C  & K \\   
              &    &    &    &   
        \end{array}
        \end{equation*}
        After unwinding the band it becomes scrambled to
        \begin{equation*}
          HENTEIDTLAEAPMRCMUAK
        \end{equation*}
        \end{block}
      \end{column}
      \begin{column}{.48\linewidth}
        \begin{block}{Caesar Shift Cipher}
          \begin{itemize}
          \item some items and $\alpha=\gamma, \sum_{i}$
          \item some items
          \item some items
          \item some items
          \end{itemize}
          $$\alpha=\gamma, \sum_{i}$$
        \end{block}

        \begin{block}{Frequency Analysis}
          \begin{itemize}
          \item some items
          \item some items
          \item some items
          \item some items
          \end{itemize}
        \end{block}

        \begin{block}{Introduction}
          \begin{itemize}
          \item some items and $\alpha=\gamma, \sum_{i}$
          \item some items
          \item some items
          \item some items
          \end{itemize}
          $$\alpha=\gamma, \sum_{i}$$
        \end{block}
      \end{column}
    \end{columns}
  \end{frame}
\end{document}


%%%%%%%%%%%%%%%%%%%%%%%%%%%%%%%%%%%%%%%%%%%%%%%%%%%%%%%%%%%%%%%%%%%%%%%%%%%%%%%%%%%%%%%%%%%%%%%%%%%%
%%% Local Variables: 
%%% mode: latex
%%% TeX-PDF-mode: t
%%% End:
