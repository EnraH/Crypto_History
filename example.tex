\documentclass[final,hyperref={pdfpagelabels=false}]{beamer}
\mode<presentation>
  {
  %  \usetheme{Berlin}
  \usetheme{USI}
  }
  %\usepackage{times}
  \usepackage{amsmath,amsthm, amssymb, amsfonts, latexsym}
  \boldmath
  \usepackage[english]{babel}
  \usepackage[latin1]{inputenc}
  \usepackage[orientation=portrait,size=a0,scale=1.4,debug]{beamerposter}
  \usepackage{tikz}
  \usetikzlibrary{calc}
  \usefonttheme{professionalfonts}

  %%%%%%%%%%%%%%%%%%%%%%%%%%%%%%%%%%%%%%%%%%%%%%%%%%%%%%%%%%%%%%%%%%%%%%%%%%%%%%%%%5
  \graphicspath{{figures/}}
  \title[Crypto History]{The History of Cryptography}
  \author[Hansen and Wolf]{Arne Hansen and Prof Stefan Wolf}
  \institute[USI]{Cryptography and Quantum Information, USI Lugano}
  \date{Jul. 31th, 2007}


  %%%%%%%%%%%%%%%%%%%%%%%%%%%%%%%%%%%%%%%%%%%%%%%%%%%%%%%%%%%%%%%%%%%%%%%%%%%%%%%%%5
  \begin{document}
  \begin{frame}{} 
    \vfill
    \begin{columns}[t]
    \begin{column}{.48\linewidth}
    \begin{block}{\large Hiding Information: A Definition of Cryptography}
      \begin{figure}
      \centering
      \begin{tikzpicture}[font=\footnotesize,
          grow=right, level 1/.style={sibling distance=6em},
          level 2/.style={sibling distance=6em}, level distance=10cm]
          \node {Hidding Information} % root
            child { node {Steganography: writing physically hidden}
            }
            child { node {\textbf{Cryptography} information hidden}
              child { node {Substitution} }
              child { node {Permutation} }
            };
      \end{tikzpicture}
      \caption{Categories of Secrecy}
      \end{figure}
      \alert{Cryptography} hidding content of a message without hiding the writing itself.
    \end{block}
    \begin{block}{\large The Scheme of Cryptography}
    \begin{figure}
      \begin{tikzpicture}[scale=0.9]
      \coordinate (plaintext1) at (0.,-1);
      \coordinate (plaintext2) at (10.,-1);
      \coordinate (key1) at (0.,1);
      \coordinate (key2) at (10,1);
      \coordinate (cipher_algorithm) at (2,0);
      \coordinate (decipher_algorithm) at (8,0);
      \coordinate (ciphertext) at (5,0);
      \filldraw[very thick,color=blue!70!black!90, fill=blue!50!black!50!] plot[smooth cycle,tension=0.05] coordinates{(3.6,-2.2) (3.6,2.2) (6.4,2.2) (6.4,-2.2)};
       \foreach \y in {-4,...,4}
                 \draw (3.8,0.5*\y)--(6.2,0.5*\y);
      \filldraw[very thick,color=red!70!black!90, fill=red!50!black!50!] plot[smooth cycle,tension=0.05] coordinates{($(plaintext1)+ (-1.4,-2.2)$) ($(plaintext1)+ (-1.4,2.2)$) ($(plaintext1)+ (1.4,2.2)$) ($(plaintext1)+ (1.4,-2.2)$)};
       \foreach \y in {-4,...,4}
                 \draw ($(plaintext1)+(-1.2,0.5*\y)$)--($(plaintext1)+(1.2,0.5*\y)$);
      \filldraw[very thick,color=red!70!black!90, fill=red!50!black!50!] plot[smooth cycle,tension=0.05] coordinates{($(plaintext2)+ (-1.4,-2.2)$) ($(plaintext2)+ (-1.4,2.2)$) ($(plaintext2)+ (1.4,2.2)$) ($(plaintext2)+ (1.4,-2.2)$)};
       \foreach \y in {-4,...,4}
                 \draw ($(plaintext2)+(-1.2,0.5*\y)$)--($(plaintext2)+(1.2,0.5*\y)$);
      %\node [label=right:$\S$] (label) at (sep) {};
      % \node [label=right:$\PPT$] (label) at (ppt) {};
      %\node [label=right:$\rho_{AB}$] (label) at (rho) {};
      %\node [label=right:$\mathcal{D}$] (label) at (dens) {};
      \end{tikzpicture}
    \end{figure}
    \end{block}
    \end{column}
    \begin{column}{.48\linewidth}
      \begin{block}{\large A battle between Cryptographers and Cryptoanalysts}
        An encrypted message remains only secret if the there is no way to break the restore the the 
      \end{block}
    \end{column}
    \end{columns}
    \vfill
    \vfill
    \begin{block}{\large Timeline: inventions}
      \begin{figure}
      \centering
      \begin{tikzpicture}

      \end{tikzpicture}
      \end{figure}
    \end{block}
    \vfill
    \begin{columns}[t]
      \begin{column}{.48\linewidth}
        \begin{block}{Introduction}
          \begin{itemize}
          \item some items
          \item some items
          \item some items
          \item some items
          \end{itemize}
        \end{block}
      \end{column}
      \begin{column}{.48\linewidth}
        \begin{block}{Introduction}
          \begin{itemize}
          \item some items and $\alpha=\gamma, \sum_{i}$
          \item some items
          \item some items
          \item some items
          \end{itemize}
          $$\alpha=\gamma, \sum_{i}$$
        \end{block}

        \begin{block}{Introduction}
          \begin{itemize}
          \item some items
          \item some items
          \item some items
          \item some items
          \end{itemize}
        \end{block}

        \begin{block}{Introduction}
          \begin{itemize}
          \item some items and $\alpha=\gamma, \sum_{i}$
          \item some items
          \item some items
          \item some items
          \end{itemize}
          $$\alpha=\gamma, \sum_{i}$$
        \end{block}
      \end{column}
    \end{columns}
  \end{frame}
\end{document}


%%%%%%%%%%%%%%%%%%%%%%%%%%%%%%%%%%%%%%%%%%%%%%%%%%%%%%%%%%%%%%%%%%%%%%%%%%%%%%%%%%%%%%%%%%%%%%%%%%%%
%%% Local Variables: 
%%% mode: latex
%%% TeX-PDF-mode: t
%%% End:
