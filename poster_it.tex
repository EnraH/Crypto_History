\documentclass[final,hyperref={pdfpagelabels=false}]{beamer}
\mode<presentation>
  {
  %  \usetheme{Berlin}
  \usetheme{USI}
  }
  %\usepackage{times}
  \usepackage{amsmath,amsthm, amssymb, amsfonts, latexsym}
  \boldmath
  \usepackage[italian]{babel}
  \usepackage[T1]{fontenc}
  \usepackage[utf8]{inputenc}
  \usepackage[orientation=portrait,size=a0,scale=0.9,debug]{beamerposter}
  \usepackage{verbatim}
  \usepackage{IEEEtrantools}
  \usepackage{tikz}
  \usetikzlibrary{calc}

  \newcommand{\gearmacro}[5]{%  
    \foreach \i in {1,...,#1} {%
      [rotate=(\i-1)*360/#1]  (0:#2)  arc (0:#4:#2) {[rounded corners=1.5pt]
        -- (#4+#5:#3)  arc (#4+#5:360/#1-#5:#3)} --  (360/#1:#2)
    }}  

  %%%%%%%%%%%%%%%%%%%%%%%%%%%%%%%%%%%%%%%%%%%%%%%%%%%%%%%%%%%%%%%%%%%%%%%%%%%%%%%%%5
  \graphicspath{{figures/}}
  \title[Crypto History]{La Storia della Crittografia}
  \author[Hansen and Wolf]{Arne Hansen e Prof Stefan Wolf}
  \institute[USI]{Crittografia e Informazione Quantistica, USI Lugano}
  \date{Jul. 31th, 2007}


  %%%%%%%%%%%%%%%%%%%%%%%%%%%%%%%%%%%%%%%%%%%%%%%%%%%%%%%%%%%%%%%%%%%%%%%%%%%%%%%%%5
  \begin{document}
  \tikzset{
    TLstegan/.style = {fill = gray!20!white,
                      draw=AHdarkblue,align= left,text width=5cm,font=\tiny,line width=2pt},
    TLcrypto/.style = {fill = gray!20!white,
                      draw=AHdarkorange,align= left,text width=5cm,font=\tiny,line width=2pt},
    TLphases/.style = {color=black,left,text width=10cm,font=\tiny,line width=0pt}
  }
  \begin{frame}{} 
    \vfill
    \vfill
    \begin{block}{\large Cronologia della Crittografia}
      \begin{figure}
      \centering
      \begin{tikzpicture}[scale=1.9]
          % define coordinates
          \coordinate (Herodotus) at (-5,0);
          \coordinate (Caesar) at (-0.6,0);
          \coordinate (scytale) at (-4.5,0);
          \coordinate (freq-analysis) at (9.5,0);
          \coordinate (Vigenere) at (15.53,0);
          \coordinate (Babbage) at (18.50,0); 
          % forall coordinates in 20th century: magnify by factor 10
          \coordinate (one-time-pad) at (1.7,0);
          \coordinate (enigma) at (1.8,0);
          \coordinate (rejewski) at (3.2,0);
          \coordinate (navajo) at (4.2,0);
          \coordinate (shannon) at (4.8,0);
          \coordinate (colossus) at (4.3,0);
          \coordinate (nsa) at (5.2,0);
          \coordinate (feistel) at (7.1,0);
          \coordinate (des) at (7.6,0);
          \coordinate (diffie-hellman) at (7.6,0);
          \coordinate (rsa) at (7.7,0);
          \coordinate (BB84) at (8.4,0);
          \coordinate (pgp) at (9.1,0); 
          % phases of cryptography
          \fill [AHdarkorange!30] ($(Caesar) + (0.0,-8)$) rectangle (30,-7);
          \draw[line width=3pt, color=AHdarkorange] (Caesar) -- ($(Caesar) + (0.0,-8)$) node[TLphases, above right] {cifrario di sostituzione monoalfabetico};
          \draw[line width=3pt, color=AHdarkorange] (Vigenere) -- ($(Vigenere) + (0.0,-8)$) node[TLphases, above right] {cifrario di sostituzione {\em poli}alfabetico};
          \draw[line width=3pt, color=AHdarkorange] ($(enigma) + (19,0)$) -- ($(enigma) + (19.0,-8)$) node[TLphases, above right] {{\em macchine} cifrarie polialfabetiche};
          % create baseline
          \draw[line width=2pt,dashed] (-10,0) -- (-5.20,0) ;
          \draw[line width=2pt] (-5.20,0) -- (32,0);
          \draw[line width=5pt] (18.8,0) -- (31.2,0);
          \coordinate (cryptography) at (-9.8,3);
          \coordinate (cryptanalysis) at (-9.8,-3);
          \draw[line width=6pt,color=black!80] (25,0) circle (6.2); 
          \node[rotate=90] (label) at (cryptography) {\small \textsc{Crittografia}};
          \node[rotate=90] (label) at (cryptanalysis) {\small \textsc{Crittoanalisi}};
          \foreach \t in {-5,...,18}{
            \draw[line width=2pt] (\t,-.2) -- (\t,.2) node [below=.5cm] {\tiny \t 00};
            \draw[line width=1pt] ($(\t,-.1) + (.5,0)$) -- ($(\t,.1) + (0.5,0)$);}
          \foreach \t in {0,...,12}{
            \draw[line width=3pt] let \n1 = {int(1900+\t*10)} in
              ($(19,0)+(\t,-.2)$) -- ($(19,0)+(\t,.2)$) node [below=.6cm] {\tiny \n1};
            \draw[line width=2pt] ($(19.5,0)+(\t,-.1)$) -- ($(19.5,0)+(\t,.1)$);}
            
          % declare nodes
          % cryptographic achievement
          \draw[line width=1pt] ($(Herodotus) + (0,0.0)$) -- ($(Herodotus) + (0,1.0)$) node [TLstegan,above left] {Lo storico greco {\em Erodoto} discute della steganografia};
          \draw[line width=1pt] ($(scytale) + (0,0.0)$) -- ($(scytale) + (0,4.0)$) node [TLcrypto,above] {\textbf{ V secolo AC} {\em Scitala}: strumento crittografico utilizzato dagli spartani};
          \draw[line width=1pt] ($(Caesar) + (0,0.0)$) -- ($(Caesar) + (0,1.0)$) node [TLcrypto,above right] {{\em Cifrario di Cesare}: crittografa attraverso lo spostamento di lettere dell'alfabeto};
          \draw[line width=1pt] ($(Vigenere) + (0,0.0)$) -- ($(Vigenere) + (0,1.0)$) node [TLcrypto,above left] {\textbf{1553} Giovan Battista Bellaso sviluppa il cifrario polialfabetico chiamato {\em cifrario di Vigen\`{e}re}};
          \draw[line width=1pt] ($(one-time-pad) + (19,0.0)$) -- ($(one-time-pad) + (19,1.0)$) node [TLcrypto,above left] {\textbf{1917} il {\em cifrario di Vernam} --- un cifrario con sicurezza assoluta --- \`{e} inventato};
          \draw[line width=1pt] ($(enigma) + (19,0.0)$) -- ($(enigma) + (19,3.0)$) node [TLcrypto,above] {\textbf{1918}{\em ENIGMA} viene inventata da Arthur Scherbius};
          \draw[line width=1pt] ($(shannon) + (19,0.0)$) -- ($(shannon) + (19,2.3)$) node [TLcrypto,above] {\textbf{1948/49} {\em Claude Shannon} getta le basi per la teoria dell'informazione e la crittografia formale};
          \draw[line width=1pt] ($(navajo) + (19,0.0)$) -- ($(navajo) + (19,1.0)$) node [TLcrypto,above] {\textbf{1942} {\em Navajos} viene utilizzato dall'esercito americano per crittografare i messaggi};
          \draw[line width=1pt] ($(feistel) + (19,0.0)$) -- ($(feistel) + (19,7.0)$) node [TLcrypto,above left] {\textbf{1971} Horst Feistel sviluppa il cifrario a blocchi {\em Lucifer} per IBM};
          \draw[line width=1pt] ($(BB84) + (19,0.0)$) -- ($(BB84) + (19,7.0)$) node [TLcrypto,above right] {\textbf{1984} Charles Bennet e Gilles Brassard inventano un protocollo per la distribuzione quantistica di chiavi, chiamato {\em BB84}};
          \draw[line width=1pt] ($(des) + (19,0.0)$) -- ($(des) + (19,4.0)$) node [TLcrypto,above left] {\textbf{1976} Una versione di Lucifer viene utilizzata come base per lo standard {\em Data Encryption Standard (DES)}};
          \draw[line width=1pt] ($(rsa) + (19,0.0)$) -- ($(rsa) + (19,4.0)$) node [TLcrypto,above right] {\textbf{1977} Ron Rivest, Adi Shamir e Leonard Aldeman sviluppano una algoritmo di crittografia asimmetrico {\em con chiave pubblica} {\em RSA} (al contrario di Diffie)};
          \draw[line width=1pt] ($(diffie-hellman) + (19,0.0)$) -- ($(diffie-hellman) + (19,0.5)$) node [TLcrypto,above] {\textbf{1976} Martin Hellman, Whitfield Diffie e Ralph Merkle pubblicano un protocollo per lo {\em scambio di chiavi di crittografia}};
          \draw[line width=1pt] ($(pgp) + (19,0.0)$) -- ($(pgp) + (19,0.5)$) node [TLcrypto,above right] {\textbf{1991} Phil Zimmermann crea un software contenente diversi algoritmi simmetrici, asimmetrici e di firma digitale chiamato {\em Pretty Good Privacy} e lo distribuisce pubblicamente};
          % cryptoanalytic achievements
          \draw[line width=1pt] ($(freq-analysis) + (0,0.)$) -- ($(freq-analysis) + (0,-1.0)$) node [TLcrypto,below left] {{\em analisi delle frequenze} di cifrari monoalfabetici};
          \draw[line width=1pt] ($(Babbage) + (0,0.)$) -- ($(Babbage) + (0,-1.0)$) node [TLcrypto,below left] {\textbf{1850s} {\em Charles Babbage} viola i cifrari polialfabetici};
          \draw[line width=1pt] ($(rejewski) + (19,0)$) -- ($(rejewski) + (19,-3.0)$) node [TLcrypto,below left] {\textbf{1932} {\em Marian Rejewski} viola Enigma};
          \draw[line width=1pt] ($(colossus) + (19,0)$) -- ($(colossus) + (19,-1.0)$) node [TLcrypto,below left] {\textbf{1943} Tommy Flower costruisce la macchina {\em Colossus} di Max Newman, il primo computer moderno};
          \draw[line width=1pt] ($(nsa) + (19,0)$) -- ($(nsa) + (19,-1.0)$) node [TLcrypto,below right] {\textbf{1952} la {\em National Security Agency (NSA)} viene fondata};
      \end{tikzpicture}
      \end{figure}
      La cronologia mostra le pi\`{u} grandi conquiste nella storia della crittografia e della crittoanalisi. Sul fondo viene delineata l'evoluzione dei cifrari di sostituzione dal quello monoalfabetico a quello polialfabetico, utilizzato oggigiorno nei computer.
    \end{block}
%%%% End of timeline %%%%%%%%%%%%%%%%%%%%%%%%%%%%%
%%%%%%%%%%%%%%%%%%%%%%%%%%%%%%%%%%%%%%%%%%%%%%%%%%
    
    \begin{columns}[t]
    \begin{column}{.3\linewidth}
    \begin{block}{Una storia segreta}
      Poter accedere ad informazioni sensibili può portare grande vantaggi, come non possederne altre può risultare altrettanto svantaggioso. Tutto ciò può avere vaste implicazioni politiche, economiche e strategiche. Ad esempio, i servizi segreti di tutto il mondo raccolgono informazioni sensibili per supportare politici ed eserciti, mentre la diffusione di segreti aziendali può portare a gravi perdite economiche. Queste motivazioni hanno stimolato la ricerca di metodi sia per celare informazioni che per scoprirne di nascoste. \par
      In particolare, i servizi segreti cercano di mantenere dei vantaggi nell'accedere o nel celare informazioni. Alcune informazioni sulla crittografia stessa sono state celate: le scoperte più importanti vengono nascoste, i relativi documenti classificati, le persone coinvolte sono costrette al silenzio. Pertanto, la storia della crittografia risulta ancora misteriosa in alcuni suoi aspetti ed occasionalmente la si è dovuta riscrivere solo dopo che alcune scoperte sono state rivelate al pubblico.
    \end{block}
      \begin{block}{L'eterna lotta tra crittografi e crittoanalisti}
        Mentre i crittografi ricercano nuovi cifrari di offuscamento delle informazioni, i crittoanalisti studiano i cifrari già noti e cercano di violarli per poter quindi accedere alle informazioni nascoste. Nella storia il vantaggio si è spesso alternato a favore di uno o dell'altro. \par 
        Con la nascita dell'analisi delle frequenze, i cifrari monoalfabetici (come il cifrario di Cesare) sono stati riconosciuti come insicuri. I crittografi, quindi, hanno introdotto i cifrari polialfabetici come ,ad esempio, il cifrario di Vigenère. Di conseguenza, l'analisi delle frequenze è stata ulteriormente rifinita fino a quando i crittoanalisti sono stati nuovamente in grado di accedere alle informazioni nascoste. Questo alternarsi di vittorie e sconfitte prosegue da allora \ldots
      \end{block}
        \begin{block}{La steganografia nel mondo antico}
          Nei suoi scritti lo storico greco Erodoto ha discusso diversi metodi per celare le informazioni. Nel V secolo AC, per nascondere i messaggi, le lettere venivano ricoperte da uno strato di cera. Un espediente alquanto originale fu quello di tatuare il messaggio direttamente sulla testa rasata del messaggero. Quando i capelli ricrebbero sulla sua testa, il messaggero finalmente potè recapitare il messaggio. Queste modalità fisiche di occultamento del messaggio sono chiamate steganografia (dal greco {\it steganos} ``nascosto, coperto'').
        \end{block}
        \begin{block}{Il cifrario di Cesare: i cifrari monoalfabetici}
          Il primo uso militare documentato della crittografia è attribuibile a Giulio Cesare. Al fine di proteggere la corrispondenza militare inviata alle truppe di Cicerone dal poter venire intercettata e letta dai nemici, Cesare rimpiazzò le lettere romane con quelle greche. Da quel momento, la crittografia e la crittoanalisi hanno avuto un ruolo fondamentale all'interno delle guerre. I servizi segreti e militari si sono focalizzati sia sullo sviluppo di cifrari, sia su metodi per violare quelli degli avversari. \par
          Cesare utilizzò anche dei cifrari a sostituzione nei quali, anzichè introdurre dei nuovi simboli, una lettera dell'alfabeto romano veniva sostituite da un'altra. Questo schema di sostituzione era la chiave sia di cifratura che di lettura dei messaggi, pertanto doveva essere mantenuto assolutamente segreto.
        \end{block}

        \begin{block}{Analisi delle frequenze}
          Una delle prime scoperte di crittoanalisi nacque nel mondo arabo dagli studi linguistici del Corano nel IX secolo. I teologi analizzarono la struttura delle sacre scritture per poter definire la loro origine. A tal fine, gli studiosi contarono le lettere contenute nei testi e studiarono la frequenza con la quale esse apparivano. Si scoprì quindi come alcune lettere fossero più utilizzate di altre. Ad esempio, in inglese la lettera più frequente è la ``e''. Contando la frequenza delle lettere in un testo cifrato, diventa quindi possibile indovinare lo schema di sostituzione utilizzato per criptare i messaggi. Dal momento in cui {\em l'analisi delle frequenze} venne definita e studiata, i crittoanalisti poterono violare qualsiasi messaggio, fino al momento in cui ulteriori metodi crittografici vennero sviluppati. Nel 1553 l'italiano Giovan Battista Bellaso suggerì di utilizzare più di un solo schema di sostituzione per crittare un messaggio, e di scambiare la loro combinazione spesso. Questi cifrari così definiti vennero chiamati cifrari polialfabetici. Nel XIX secolo il britannico Charles Babbage studiò come rifinire l'analisi delle sequenze per poter violare anche i cifrari polialfabetici.
        \end{block}

    \end{column}
    \begin{column}{.3\linewidth}
        \begin{block}{Il blocco monouso: la sicurezza perfetta}
          Il blocco monouso --- un cifrario con una chiave casuale lunga tanto quanto il messaggio stesso --- venne (re-)inventato nel 1917 (dopo essere stata descritto per la prima volta nel 1882). Se la chiave rimane segreta e il blocco monouso viene utilizzato una e una sola volta (come il nome stesso suggerisce) questo cifrario è assolutamente sicuro, come venne dimostrato da Shannon nel 1945. Il grande svantaggio di questo meccanismo è il continuo scambio di chiavi tanto lunghe quanto il messaggio che deve avvenire senza che esse vengano scoperte o manomesse. \par
          La linea di comunicazione diretta tra Mosca e Washington D.C., istituita dopo la crisi missilistica cubana del 1963, era resa sicura dall'utilizzo di un blocco monouso. \par
          Infine, questa tecnica ha anche un'importanza teoretica in teoria dell'informazione ed in crittografia.
        \end{block}
        \begin{block}{Enigma: la prima macchina di cifratura}
          Nel 1918 il tedesco Arthur Scherbius inventò la macchina di cifratura Enigma. Nel primo periodo dopo la sua commercializzazione la Enigma non ottenne il successo sperato, per lo più a causa del suo costo elevato. Tuttavia, nel 1923 Churchhill rivelò come durante la prima guerra mondiale i metodi di crittografia tedeschi vennero regolarmente decifrati. L'Alleanza riuscì ad ottenere informazioni vitali dalla decifratura dei messaggi, senza che l'esercito tedesco riuscisse a comprendere che i propri sistemi fossero compromessi. (Ad esempio, la decifratura del ``telegramma Zimmermann'' da parte dell'esercito britannico accelerò l'ingresso nel conflitto degli Stati Uniti d'America.) \par
          Comprendendo quanto antiquati e compromessi fossero i sistemi di crittografia utilizzati nella prima guerra mondiale, la Germania decise di acquistare la macchina Enigma e di utilizzarla come metodo crittografico ufficiale, affidandosi al sistema crittografico più avanzato dell'epoca. Tuttavia, anche la Enigma si dimostrò vulnerabile. Nel 1932 il matematico polacco Marian Rejewski riuscì a decifrare dei messaggi crittografati dalla macchina Enigma. Tramite una spia, il matematico riuscì ad ottenere informazioni su come la macchina funzionava. La spia rivelò, inoltre, che la chiave del messaggio veniva trasmessa all'inizio del messaggio ed era inviata sempre due volte per evitare che vi fossero errori di trasmissione. Queste informazioni furono sufficienti per permettere a Rejewski di violare il meccanismo di crittografia della Enigma. Nel 1939 --- poco prima che i tedeschi attaccassero la Polonia --- l'esercito tedesco decise di incrementare la sicurezza della Enigma. Rejewski non riuscì più quindi a violare le comunicazioni tedesche. Nell'agosto dello stesso anno, un solo mese prima dell'inizio della seconda guerra mondiale, i polacchi riuscirono ad inviare segretamente le informazioni sulla Enigma in loro possesso a francesi e inglesi, che ancora assumevano la totale inviolabilità della macchina. \par
          Negli anni successivi, gli inglesi crearono il proprio gruppo di crittoanalisti con sede a Bletchley Park. I crittoanalisti, tra cui Alan Turing, riuscirono  a decrifrare regolarmente le comunicazioni tedesche. I tedeschi, come già successo, non sospettarono che il proprio metodo di crittografia fosse compromesso. Le informazioni ottenute dagli inglesi furono estremamente importanti dal punto di vista strategico e la guerra sarebbe potuta terminare in modo completamente diverso se solo gli Alleati non fossero stati in grado di decifrare i messaggi tedeschi. \par
          Fu proprio a Bletchley Park che Max Newman creò il primo computer progettato appositamente per violare i cifrari della Enigma.\par
          La vulnerabilità della Enigma fu un segreto ben custodito dall'Alleanza in quanto la possibilità di leggere i messaggi tedeschi era garantita fin tanto che la macchina era da loro considerata inviolabile.
        \end{block}
        \begin{block}{Il codice Navajo: un codice `linguistico' inviolato}
          Durante la seconda guerra mondiale, cifrari come quello dell'Enigma erano comunemente utilizzati. Lo svantaggio più grande di questi cifrari risiede nello sforzo richiesto per cifrare e decifrare i messaggi. Nelle situazioni più critiche, che richiedevano uno scambio rapido di messaggi cifrati, questi meccanismi venivano abbandonati, rivelando direttamente il contenuto ai nemici in ascolto sulle frequenze radio. Per questo motivo nel 1942 Philip Johnston, un ingegnere americano, suggerì di tradurre i messaggi nel linguaggio tribale dei Navajo prima di trasmetterlo. Dal momento che sia la grammatica che il vocabolario non avevano similitudini ne' con le lingue europee ne' come le lingue asiatiche, si rivelò un cifrario estremamente sicuro. Numerosi membri della tribù Navajo vennero reclutati per svolgere il servizio crittografico e, se durante la guerra molte macchine di cifratura vennero violate, la lingua Navajo non lo fu mai.
        \end{block}

    \end{column}
    \begin{column}{.3\linewidth}
        \begin{block}{Lucifer: l'inizio dei cifrari a blocchi}
          Nel 1934 Horst Feistel si trasferì dalla Germania agli Stati Uniti. Dopo il termine del secondo conflitto mondiale, quando entrò nel campo di ricerca della crittografia, la NSA non approvò le sue ricerche in quanto avrebbe potuto privare l'organizzazione dall'accesso ad informazioni sensibili. Negli anni '70, mentre lavorava per l'IBM, Feistel sviluppò un algoritmo crittografico chiamato Lucifer --- un cifrario a blocco pubblicizzato come il migliore e più sicuro metodo crittografico in commercio. Da quel momento Lucifer venne adottato come nucleo dello standard di crittografia dei dati (o Data Encryption Standard, DES) a partire dal 1976. Voci dicono che la NSA cercò di interferire all'adozione dello standard al fine di indebolirne l'utilizzo. Ad oggi, i cifrari a blocchi sono il cavallo di battaglia della crittografia e vengono utilizzati per crittografare grandi moli di dati.
        \end{block}

        \begin{block}{Diffie-Hellman: il protocollo per lo scambio delle chiavi}
          Fino a questo momento della storia, le comunicazioni sicure hanno fatto affidamento su una chiave segreta non nota agli avversari. Scambiarsi le chiavi segrete è un'attività costosa e che richiede notevole tempo in quanto o le due parti coinvolte si incontrano di persona, oppure devono fare affidamento a degli agenti di fiducia che possano svolgerlo al loro posto. Quindi sorge spontanea la domanda: è possibile scambiarsi una chiave utilizzando un semplice canale di comunicazione pubblico? La risposta è: sì. Nel 1976 Martin Hellman, Whitfield Diffie e Ralph Merkle hanno pubblicato un protocollo per lo scambio delle chiavi.
        \end{block}

        \begin{block}{RSA: crittografia asimmetrica}
          Nela 1977 Ron Rivest, Adi Shamir e Leonard Aldeman svilupparono un protocollo di crittografia con chiavi pubbliche, realizzando l'idea abbozzata in precedenza da Diffie. Supponiamo che Alice voglia mandare segretamente un messaggio a Bob. Per far ciò, Alice prende la chiave {\em pubblica} di Bob, che è nota a tutti, ed la utilizza per crittografare il proprio messaggio. A questo punto Alice invia il messaggio crittografato a Bob. Bob usa la sua chiave {\em privata} per decrittare il messaggio. In modo più figurato: tutti hanno una chiave per poter chiudere una cassetta al cui interno depositare il messaggio segreto. Tuttavia, solo Bob possiede la chiave per poterla riaprire. \par
          In sostanza, Alice e Bob non devono scambiarsi una chiave comune come accadeva in precedenza. Alice deve solamente prendere la chiave pubblica di Bob da una fonte di fiducia ed utilizzarla. \par
          Oggigiorno per crittografare un messaggio si utilizza un cifrario asimmetrico che viene utilizzato per lo scambio di una chiave comune e successivamente su di un cifrario a blocchi per crittografare una grande quantità di dati. Grazie a Phil Zimmermann, gli algoritmi di crittografia sono oggigiorno di pubblico dominio. Zimmermann ha incontrato diversi ostacoli prima di poter rilasciare pubblicamente il suo insieme di programmi di crittografia chiamato {\em Pretty Good Privacy}: la legislazione americana, infatti, vieta l'esportazione di prodotti crittografici al di fuori del suolo statunitense. Il progetto GNU Privacy Guard offre un'implementazione open source di questi algoritmi compatibile con la maggior parte dei programmi email e chat, oltre ad un insieme di strumenti per crittografare i dati direttamente sui dischi di vari dispositivi.
        \end{block}

        \begin{block}{Bennet e Brassard: distribuzione della chiave quantistica}
          Gli algoritmi descritti fino ad ora possono essere eseguiti su un computer o un telefonino. La sicurezza che è possibile ottenere con questi algoritmi nasce dall'innumerevole varietà di schemi di sostituzione che un ipotetico attaccante dovrebbe necessariamente controllare e che surclassa attualmente la potenza computazionale disponibile. I cifrari più recenti si basano su funzioni matematiche ancor più difficili da calcolare. \par
          Charles Bennet e Gilles Brassard hanno sviluppato un protocollo per la distribuzione di chiavi chiamato {\em BB84} che è basato su un computer quantistico. Il protocollo non solo permette ad Alice e Bob di concordare segretamente una chiave, ma anche di determinare se il loro canale di comunicazione è stato compromesso o meno. Il protocollo si basa sulla meccanica quantistica. Dal momento che misurare un sistema quantistico altera il sistema stesso, una possibile spia lascerebbe delle tracce del suo tentativo di ascoltare lo scambio di messaggi. \par
          Quindi, Alice e Bob possono scambiarsi una chiave utilizzando il protocollo {\em BB84} e quindi crittografare i loro messaggi con un cifrario a blocchi monouso per aumentare ancor più il grado di sicurezza, indipendentemente dalla potenza computazionale che eventuali spie possono possedere. \par
          I computer quantistici non solo possono incrementare il grado di sicurezza della crittografia ma anche violare i cifrari più comuni.
          I protocolli classici descritti in precedenza sono basati su alcune funzioni matematiche la cui proprietà principale è quella di essere difficilmente invertibili. Tuttavia, per alcune di queste funzioni esiste un protocollo quantistico in grado di calcolarne l'inversa in modo efficiente. Al momento non esistono ancora computer quantistici efficienti, e quindi oggi i pericoli più comuni da affrontare sono la mancanza di crittografia, le implementazioni sbagliate, gli attacchi di tipo side-channel ed il furto delle chiavi crittografiche.
        \end{block}
      \end{column}
    \end{columns}
  \end{frame}
\end{document}


%%%%%%%%%%%%%%%%%%%%%%%%%%%%%%%%%%%%%%%%%%%%%%%%%%%%%%%%%%%%%%%%%%%%%%%%%%%%%%%%%%%%%%%%%%%%%%%%%%%%
%%% Local Variables: 
%%% mode: latex
%%% TeX-PDF-mode: t
%%% End:
